\documentclass[12pt, a4paper,
%oneside,      %% -- odkomentujte, pokud chcete svou práci mít pouze jednostrannou, mezera pro hřbet pak automaticky bude pouze na levé straně
%twoside,        %% -- pro oboustranné práce, mezera pro hřbet následně střídá strany.
openright
]{report}

%% Nutné balíčky a nastavení
%%%%%%%%%%%%%%%%%%%%%%%%%%%%

%% Proměnné
\newcommand\obor{INFORMAČNÍ TECHNOLOGIE} %% -- napiš číslo a název tvého oboru
\newcommand\kodOboru{18-20-M/01} %% -- napiš číslo a název tvého oboru
\newcommand\zamereni{se zaměřením na počítačové sítě a programování} %% -- napiš číslo a název tvého oboru
\newcommand\skola{Střední škola průmyslová a umělecká, Opava} %% vyplň název školy
\newcommand\trida{IT4} %% vyplň jméno svého konzultanta
\newcommand\jmenoAutora{Matěj Kotrba}  %% vyplň své jméno
\newcommand\skolniRok{2023/24} %% vyplň rok
\newcommand\datumOdevzdani{1. 1. 2024} %% vyplň rok
\newcommand\nazevPrace{Effio - webová aplikace pro vytváření testů} %% vyplň název své práce

\title{\nazevPrace} %% -- Název tvé práce
\author{\jmenoAutora} %% -- tvé jméno
\date{\datumOdevzdani} %% -- rok, kdy píšeš SOČku

\usepackage[top=2.5cm, bottom=2.5cm, left=3.5cm, right=1.5cm]{geometry} %% nastaví okraje, left -- vnitřní okraj, right -- vnější okraj

\usepackage[czech]{babel} %% balík babel pro sazbu v češtině
\usepackage[utf8]{inputenc} %% balíky pro kódování textu
\usepackage[T1]{fontenc}
\usepackage{cmap} %% balíček zajišťující, že vytvořené PDF bude prohledávatelné a kopírovatelné
\usepackage{indentfirst}

\usepackage{graphicx} %% balík pro vkládání obrázků

\usepackage{subcaption} %% balíček pro vkládání podobrázků

\usepackage{hyperref} %% balíček, který v PDF vytváří odkazy

\linespread{1.25} %% řádkování
\setlength{\parskip}{0.5em} %% odsazení mezi odstavci


\usepackage[pagestyles]{titlesec} %% balíček pro úpravu stylu kapitol a sekcí
\titleformat{\chapter}[block]{\scshape\bfseries\LARGE}{\thechapter}{10pt}{\vspace{0pt}}[\vspace{-22pt}]
\titleformat{\section}[block]{\scshape\bfseries\Large}{\thesection}{10pt}{\vspace{0pt}}
\titleformat{\subsection}[block]{\bfseries\large}{\thesubsection}{10pt}{\vspace{0pt}}


\usepackage{tocloft} % Balíček umožní přizpůsobit vzhled tabulky obsahu
\setlength{\cftbeforechapskip}{0pt}  % Menší rozestup pro kapitoly
\setlength{\cftbeforesecskip}{0pt}   % Menší rozestup pro sekce

\setcounter{secnumdepth}{2}
\setcounter{tocdepth}{1}
\usepackage{fancyhdr}
\pagestyle{fancy}
\renewcommand{\headrulewidth}{0.025pt}

\usepackage{booktabs}

\usepackage{url}

%% Balíčky co se můžou hodit :) 
%%%%%%%%%%%%%%%%%%%%%%%%%%%%%%%

\usepackage{pdfpages} %% Balíček umožňující vkládat stránky z PDF souborů, 

\usepackage{upgreek} %% Balíček pro sazbu stojatých řeckých písmen, třeba u jednotky mikrometr. Například stojaté mí: \upmu, stojaté pí: \uppi

\usepackage{amsmath}    %% Balíčky amsmath a amsfonts 
\usepackage{amsfonts}   %% pro sazbu matematických symbolů
\usepackage{esint}     %% pro sazbu různých integrálů (např \oiint)
\usepackage{mathrsfs}
\usepackage{helvet} % Helvet font
\usepackage{mathptmx} % Times New Roman
\usepackage{Oswald} % Oswald font
\usepackage{float}

%% makra pro sazbu matematiky
\newcommand{\dif}{\mathrm{d}} %% makro pro sazbu diferenciálu, místo toho
%% abych musel psát '\mathrm{d}' mi stačí napsat '\dif' což je mnohem 
%% kratší a mohu si tak usnadnit práci

\usepackage{listings}
\usepackage{xcolor}

\renewcommand{\lstlistingname}{Kód}% Listing -> Algorithm
\renewcommand{\lstlistlistingname}{Seznam programových kódů}% List of Listings -> List of Algorithms

%% Definice 
\lstdefinelanguage{JavaScript}{
	morekeywords=[1]{break, continue, delete, else, for, function, if, in,
		new, return, this, typeof, var, void, while, with},
	% Literals, primitive types, and reference types.
	morekeywords=[2]{false, null, true, boolean, number, undefined,
		Array, Boolean, Date, Math, Number, String, Object},
	% Built-ins.
	morekeywords=[3]{eval, parseInt, parseFloat, escape, unescape},
	sensitive,
	morecomment=[s]{/*}{*/},
	morecomment=[l]//,
	morecomment=[s]{/**}{*/}, % JavaDoc style comments
	morestring=[b]',
	morestring=[b]"
}[keywords, comments, strings]


\lstdefinelanguage[ECMAScript2015]{JavaScript}[]{JavaScript}{
	morekeywords=[1]{await, async, case, catch, class, const, default, do,
		enum, export, extends, finally, from, implements, import, instanceof,
		let, static, super, switch, throw, try},
	morestring=[b]` % Interpolation strings.
}

\lstalias[]{ES6}[ECMAScript2015]{JavaScript}

% Nastavení barev
% Requires package: color.
\definecolor{mediumgray}{rgb}{0.3, 0.4, 0.4}
\definecolor{mediumblue}{rgb}{0.0, 0.0, 0.8}
\definecolor{forestgreen}{rgb}{0.13, 0.55, 0.13}
\definecolor{darkviolet}{rgb}{0.58, 0.0, 0.83}
\definecolor{royalblue}{rgb}{0.25, 0.41, 0.88}
\definecolor{crimson}{rgb}{0.86, 0.8, 0.24}


\lstdefinestyle{JSES6Base}{
	backgroundcolor=\color{white},
	basicstyle=\ttfamily\footnotesize,
	breakatwhitespace=false,
	breaklines=true,
	captionpos=b,
	columns=fullflexible,
	commentstyle=\color{mediumgray}\upshape,
	emph={},
	emphstyle=\color{crimson},
	extendedchars=true,  % requires inputenc
	fontadjust=true,
	frame=single,
	identifierstyle=\color{black},
	keepspaces=true,
	keywordstyle=\color{mediumblue},
	keywordstyle={[2]\color{darkviolet}},
	keywordstyle={[3]\color{royalblue}},
 literate=%
{á}{{\'a}}1 {č}{{\v{c}}}1 {ď}{{\v{d}}}1 {é}{{\'e}}1 {ě}{{\v{e}}}1
{í}{{\'i}}1 {ň}{{\v{n}}}1 {ó}{{\'o}}1 {ř}{{\v{r}}}1 {š}{{\v{s}}}1
{ť}{{\v{t}}}1 {ú}{{\'u}}1 {ů}{{\r{u}}}1 {ý}{{\'y}}1 {ž}{{\v{z}}}1,		
	numbers=left,
	numbersep=5pt,
	numberstyle=\tiny\color{black},
	rulecolor=\color{black},
	showlines=true,
	showspaces=false,
	showstringspaces=false,
	showtabs=false,
	stringstyle=\color{forestgreen},
	tabsize=2,
	title=\lstname,
	upquote=true  % requires textcomp
}

\lstdefinestyle{JavaScript}{
	language=JavaScript,
	style=JSES6Base,
}
\lstdefinestyle{ES6}{
	language=ES6,
	style=JSES6Base
}

\usepackage{lipsum} %% balíček který píše lipsum (nesmyslný text, který se používá pro kontrolu typografie)

%% Začátek dokumentu
%%%%%%%%%%%%%%%%%%%%
\begin{document}
	
	\pagestyle{empty}
	\pagenumbering{Roman}
	
	\cleardoublepage

%% Titulní stránka s informacemi
%%%%%%%%%%%%%%%%%%%%%%%%%%%%%%%%%%%%%%%%
	
	{\fontfamily{phv}\selectfont
		%% Logo školy
		\begin{figure}[h]
			\centering
			\includegraphics[width=0.6\linewidth]{image/logo-skoly.png} 
		\end{figure}
		
		
		%% Hlavička práce a její název (viz proměnná \nazev prace)
		%% \sffamily %%% bezpatkové písmo - sans serif
		{\bfseries %%% písmo na stránce je tučně
			\begin{center}
				\vspace{0.025 \textheight}
				\LARGE{ZÁVĚREČNÁ STUDIJNÍ PRÁCE}\\
				\large{dokumentace}\\
				\vspace{0.075 \textheight}
				\LARGE {\nazevPrace}\\
			\end{center}  
		}%%%
		
		\begin{figure}[h]
			\centering
			\includegraphics[width=0.8\linewidth]{image/effio.png} 
		\end{figure}
		
		\vspace{0.02 \textheight}
		\begin{table}[h!]
			\begin{tabular}{ll}
				\textbf{Autor:} & \jmenoAutora\\ 
				\textbf{Obor:} & \kodOboru { } \obor\\
				\textbf{} & \zamereni\\
				\textbf{Třída:} & \trida\\
				\textbf{Školní rok:} & \skolniRok\\
			\end{tabular}
			
		\end{table}		
	}
	
\cleardoublepage %% Zalomení dvojstránky
	
%% Stránka obsahující poděkování a prohlášení
%%%%%%%%%%%%%%%%%%%%%%%%%%%%%%%%%%%%%%%%%%%%%%%%%%%%%%%%

%% Poděkování - nepovinné
%%%%%%%%%%%%%%%%%%%%%%%%%%%%
	
	\noindent{\large{\bfseries{Poděkování}\\}}
	\noindent Rád bych poděkoval Mgr. Markovi Lučnému za poskytuntí konzultace ohledně tohoto projektu.
	
	\vspace*{0.7\textheight} %% Vertikální mezeru je možné upravit

%% Prohlášení - povinné
%%%%%%%%%%%%%%%%%%%%%%%%%%%%
	\noindent{\large{\bfseries{Prohlášení}\\}}  %% uprav si koncovky podle toho na jaký rod se cítíš, vypadá to pak lépe :) 
	\noindent{Prohlašuji, že jsem závěrečnou práci vypracoval samostatně a uvedl veškeré použité 
		informační zdroje.\\}
	\noindent{Souhlasím, aby tato studijní práce byla použita k výukovým a prezentačním účelům na Střední průmyslové a umělecké škole v Opavě, Praskova 399/8.}
	\vfill
	\noindent{V Opavě \datumOdevzdani\\}
	\noindent
	\begin{minipage}{\linewidth}
		\hspace{9.5cm} 
		\begin{tabular}{@{}p{6cm}@{}}
			\dotfill \\
			Podpis autora
		\end{tabular}
	\end{minipage}
	
	\cleardoublepage %% Zalomení dvojstránky

%% Stránka obsahující abstrakt (anotaci)
%%%%%%%%%%%%%%%%%%%%%%%%%%%%%%%%%%%%%%%%%%%%%%%%%%%%%%%%	

%% Abstrakt v češtině
%%%%%%%%%%%%%%%%%%%%%%%%%%%%
	\noindent{\Large{\bfseries{Anotace}\\}}
	\noindent Výsledkem projektu je funkční webová aplikace pro vytváření a vyplňování testů, které se skládají z~různých možností otázek včetně programovací. Aplikace zahrnuje přihlášení přes Google a GitHub. Uživatel vytváří jednotlivé testy výběrem šablony, nebo importem z GIFT formátu, otázek, komentářů a následně upravuje detaily testu jako jméno, popis, obrázek, upravitelný známkovací systém, zařazení do skupin a tagy. Hotový test může sám zkusit z~vlastní kolekce testů a nebo z~komunitního centra, kde se nacházejí komunitou vytvořené testy, po vyplnění testu se uživateli objeví výsledky a známka. Kromě výroby a vyplňování testů aplikace obsahuje také skupiny, kde mohou mezi sebou uživatelé komunikovat pomocí chatu a majitel do ní může sdílet testy, na dříve vyplněné testy se může podívat v sekci testové historie. Přehled o aktivitě uživatele si může prohlédnout v dashboardu prostřednictvím vizuálních grafů. Dříve vytvořené testy se dají v části kolekce editovat, mazat a také exportovat do GIFT formátu v textovém souboru pro použití například v~Moodlu. Aplikace disponuje zcela responsivním designem se světlým a tmavým režimem.
	
	\vspace{18pt}
	
	\noindent{\large{\bfseries{Klíčová slova}}}
	
	\noindent webová aplikace, databáze, responsivní design, účty, grafy, tvorba testů, barevné režimy
	
	\vspace{18pt}


	
	\clearpage %% Zalomení stránky
	\noindent{\Large{\bfseries{Abstract}\\}}
	\noindent The result of the project is a functional web application for creating and filling out tests, which consist of various types of questions, including programming ones. The application includes login via Google and GitHub. Users can create individual tests by selecting a template or importing from the GIFT format. They can add questions, comments, and then edit test details such as name, description, image, editable grading system, grouping, and tags. A completed test can be taken from their own collection or from the community center, where tests created by the community are available. After completing a test, users can view the results and grade. In addition to test creation and completion, the application also includes groups where users can communicate through a chat, and the owner can share tests. Previously completed tests can be reviewed in the test history section. Users can check their activity overview in the dashboard through visual graphs. Previously created tests can be edited, deleted, and exported to GIFT format in a text file for use, for example, in Moodle. The application features a fully responsive design with both light and dark modes.
	
	\vspace{18pt}
	
	\noindent{\large{\bfseries{Keywords}}}
	
	\noindent web application, database, responsive design, user accounts, graphs, test creation, color modes
	
	\vspace{18pt}

%% Stránka s generovaným obsahem
%%%%%%%%%%%%%%%%%%%%%%%%%%%%%%%%%%%%%%%	
	
	\tableofcontents %% Vygeneruje tabulku s obsahem

	\pagenumbering{arabic} %% Nastavení způsobu číslování stránek (alternativy roman | Roman)
	\setcounter{page}{1} %% Nastavení počitadla stránek

%% Stránka s úvodem - povinná část
%%%%%%%%%%%%%%%%%%%%%%%%%%%%%%%%%%%%%%%		
	\chapter*{Úvod}
	\label{intro}
%Tento příkaz vytvoří novou kapitolu s názvem "Úvod" ve vašem dokumentu.
%Hvězdička * u příkazu \chapter* znamená, že tato kapitola nebude mít číslo. Ve výsledném dokumentu se tedy objeví jako "Úvod" bez předcházejícího čísla kapitoly, které se obvykle zobrazuje u číslovaných kapitol.
%Tento příkaz také znamená, že kapitola se automaticky neobjeví v obsahu, protože LaTeX standardně zahrnuje do obsahu pouze číslované kapitoly.
	\addcontentsline{toc}{chapter}{Úvod}
%Tento příkaz ručně přidává záznam do obsahu.
%První parametr toc označuje, že přidáváme záznam do Table of Contents (obsahu).
%Druhý parametr chapter specifikuje úroveň záznamu. V tomto případě říkáme, že přidávaný záznam má být považován za kapitolu.
%Třetí parametr Úvod je text, který se objeví v obsahu. V tomto případě bude v obsahu zobrazen název "Úvod".	
V dnešní době se běžně využívají webové aplikace, které umožňují vytváření testů/kvízů, které poté jiní uživatelé vyplňují. Prostředí těchto aplikací jsou však často nepřehledné a vytváření testů či kvízů je úporné. S touto myšlenkou jsem se rozhodl vytvořit aplikaci, která by kombinovala možnosti jiných aplikací s přehledným moderním zobrazením a dalšími užitečnými prvky.

Má aplikace by kromě již zmíněné funkcionality pro tvorbu testů a kvízů měla do jisté míry umožňovat prvky sociálních sítí jako třeba skupiny, komunitní místo kde by se mimo jiné zobrazovaly testy ostatních uživatelů. Hlavní myšlenkou bylo vytvořit nejen aplikaci jako takovou ale také využít moderní technologie a postupy, neboli vytvořit ji \uv{typesafe}, bez potřeby vlastního serveru za pomocí cloudové technologie \uv{serverless} a plně responzivní pro uživatele na kterémkoli zařízení.

V dokumentaci jsou popsané využité technologie, postupy a jednotlivé funkcionality celé aplikace. První část popisuje architekturu a přístup k~řešení, následuje popis backend a frontendu, ve 4. části kapitole se potom zmiňuji o různých možnostech, které Effio nabízí. Nakonec se poohlížím na dosažené cíle a možné vylepšení.

%Tipy k psaní úvodu
%Je povinný, nadpis neměňte, rozsah - max. 1 strana. 
%Tato část práce obsahuje: 
%* náhled do řešené problematiky, zdůvodnění volby problematiky, 
%* předem definované cíle práce, 
%* motivaci pro další čtení textu včetně stručného uvedení obsahu následujících kapitol 


\chapter{Architektura a koncepty aplikace}

\section{Architektura}

\subsection{Možnosti řešení}
Pro vytvoření webové aplikace je možno využít mnoho postupů, proto zmíním několik variant nad kterými jsem uvažoval s~jejich klady se zápory

\begin{itemize}
\item Tradiční web server - jedná se o nejběžnější variantu jak vytvářet webové stránky. Jednotlivé stránky jsou vytvářeny na serveru, následně jsou poslány na klienta, poslaný kód může obsahovat také JavaScript pro frontend funkcionalitu, pro backend je možné využít jazyk dle výběru. Pro veškeré přesměrování a další akce je nutné se obrátit na server.

\item Single page application - toto řešení v~podstatě odstraňuje server a nechává veškerou zodpovědnost frontendovém frameworku jako např. React, Svelte nebo Solid, toto řešení ale nemá k~dispozici žádný způsob jak spouštět kód, která na klientu pouštět nemůžeme jako například SQL dotazy. Další nevýhodou je to, že stránka je generovaná až na klientovi JavaScriptem, proto search enginy nejsou schopny detekovat obsah stránky, což vede k~mnohem nižším výsledkům CEO.

\item Serverless - je architektura kdy vývojář využívá server poskytovatele, o ten se nemusí starat, a je škálován podle potřeby. Tento koncept ale má i své nevýhody, omezená doba relace odpovědi, menší úložný prostor pro načtení knihoven nebo třeba nemožnost spravovat svůj server.

\item Edge runtime - tato technologie je podobná Serverless architektuře, serverový kód ale neběží v jedné lokalitě ale na jednotlivých CDN. Funkce se spouštějí ne skrze Node ale přes Edge runtime, toto obsahuje svou nevýhodu, Edge není Node, a proto nemůžeme používat Node moduly, jako je třeba \uv{fs}. Další nevýhodou je velice malé množství paměti, které je pro dostupnou instanci dostupné. Výhodou je poté velice nízká cena spuštění takové funkce a bezkonkurenční rychlost odpovědi, tato výhoda je největší u serverových úkolů jako přesměrování, cookies nebo geograficky založených údajů, v~případě několikanásobných dotazů do databáze se ale cesta potřebná k~získání dat zvětšuje a výhoda rychlosti mizí.

\begin{figure}[h!]
	\centering %% příkaz, který ti obrázek zarovná na střed
	\includegraphics[width=0.7\linewidth]{image/edge-serverless.png} 
	\caption{Rozdíl mezi serverless a edge.} %% popisek obrázku, nezapomeň na citace!
	\label{fig:edge-serverless} %% označení až budeš chtít na obrázek odkazovat
\end{figure}

\item Metaframework je technologie, která kombinuje výhody \uv{single page application} a tradičního web serveru. Disponuje možností běhu kódu na serveru, přesměrováním na klientské části a dalšími výhodami. Takových technologií existuje celá řada, jako například populární NextJS, já si pro svůj projekt zvolil SvelteKit. Další fází této architektury je hostování, nejlepší variantou většinou bývá hostování přes providery jako Vercel nebo Netlify, tato architektura se poté spíše primárně aplikuje se \uv{serverless}.
\end{itemize}

Jednou z hlavních myšlenek bylo hostování Effia na cloudových službách, proto jsem si vybíral hlavně mezi technologiemi \uv{serverless} a \uv{edge computing}, výhodou providera, kterého jsem si vybral - Vercel je, že kombinace těchto technologií je velice snadná, základní variantou je \uv{serverless} s~jednoduchým přepnutím dané cesty na \uv{edge}. Ve spojení s~konceptem metaframeworku nakonec utváří velmi flexibilní, rychlou a příjmenou variantu.

\begin{figure}[h!]
	\centering %% příkaz, který ti obrázek zarovná na střed
	\includegraphics[width=1\linewidth]{image/edge-adapter.png} 
	\caption{Možnost přepnutí dané cesty ze Serverless na Edge. \cite{SvelteKit}} %% popisek obrázku, nezapomeň na citace!
	\label{fig:edge-adapter} %% označení až budeš chtít na obrázek odkazovat
\end{figure}

%Vzhledem k tomu, že záměrem bylo vytvořit aplikaci, která bude pro veškeré \uv{backendové} úlohy využívat cloud, tak je architektura relativně složité téma, proto se pokusím stručně popsat základní proces vykreslení stránky. Uživatel získává statické soubory ze \uv{CDN}, dynamicky vytvořené stránky pro první zobrazení (\uv{first render}) jsou generovány na instancích \uv{serverless funkcí} třetí strany, každá instance se vytváří podle potřeby na reálném serveru poskytovatele, není dedikovaná a ani nezůstavá aktivní po delší dobu. Server je dále využíván pro potřeby klienta, které si on sám nemůže obstarat především z bezpečnostních důvodů (DB queries, secret env variables...). Databáze vytváří spojení se serverem a ten získává potřebná data.

\section{Typesafety}

Webové aplikace standardně využívají JavaScript, ten ale obnáší signifikantní nevýhodu v podobě nemožnosti \uv{otypovat} kód, to způsobuje obtíž orientovat se v kódu, velké množství produkčních chyb a také spoustu času stráveného pochopením dříve napsaného kódu. Pro Effio jsem se tedy rozhodl využít moderní technologie a vytvořit tak téměř plně \uv{typesafe} (otypovanou) aplikaci. TypeScript v Effiu nahrazuje JavaScript, ten do tohoto jazyka přináší typy. To ale nestačí, API endpointy, stejně jako databázové dotazy stále nemůžou být otypované a proto jsem připojil také knihovny tRPC a Prisma.

\chapter{Backend}

\section{Založení a konfigurace projektu}
Prvním krokem bylo založení projektu a stažení potřebných knihoven technologií, které jsem plánoval využít. Kombinace mnou vybraných technologií nebyla kompletně konvenční a proto jsem se musel v~některých případech obrátit na komunitou vytvořené adaptéry, příkladem je například knihovna \texttt{trpc-sveltekit}, která propojuje SvelteKit a tRPC, které je primárně navrženo buď jako samostatný server a nebo jako implementace do Next.js.

\section{Architektura backendu}
\begin{figure}[h!]
	\centering %% příkaz, který ti obrázek zarovná na střed
	\includegraphics[width=1\linewidth]{image/effio-architecture-simple.png} 
	\caption{Jednoduchý přehled backendové části Effia.} %% popisek obrázku, nezapomeň na citace!
	\label{fig:effio-architecture} %% označení až budeš chtít na obrázek odkazovat
\end{figure}

	\begin{itemize}
		\item CDN - neboli Content Delivery Network je síť serverů, které jsou charakteristické hlavně tím, že jsou rozmístěny po celém světě a ve velkém množství, starají se o distribuci statického obsahu díky čemu nemusí klient data získávat ze vzdáleného server ale právě z~tohoto, který je ve většině případů mnohem blíže. Dále se starají o cachování dat, což opět zkracuje dobu pro uživatele aby zobrazil obsah.
		\item Server - neoznačuje server jako takový ale spíše místo kde se spouštějí instance serverových funkcí, což jsou funkce, které se vytváří podle potřeby na reálných serverech, které ale spravuje provider této služby, v~mém případě Vercel, respektive Cloudflare.
		
		Programátora nemusí tyto serveru vůbec zajímat, instance se sami škálují a obecně jsou pro vývojáře velice příjemných řešením. V poslední letech se tato architektura těší významné oblibě nejen malých projektů ale také velikých firem jako je Amazon, ty se ale v poslední době opět začínají vracet k~hostování vlastních serverů, které se po dlouholetých diskuzích znovu zdají výhodnější pro takto masivní účely typu Amazonu.
		
		Tento server je v Effiu využíván hlavně pro první zobrazení stránky metodou SSR (\uv{server side rendering}) a také získáváním dat, které nemohou být získány na klientu (např. SQL dotazy), pro jednotlivé stránky, formulářovými akcemi nebo jako API endpointy.
		
		\item DB - MySQL databáze hostovaná přes službu Planetscale, ta disponuje velice zajímavými možnosti jako jsou \uv{větve} podobné verzovacímu systému Git, \uv{Deploy requesty}, které zlepšují práci v~týmu nebo možnost distribuce \uv{read-only} instancí databáze do rozdílných regionů. Pro mě byla veliká výhoda rychlost a velice štědrý \uv{free tier}.
		\item Pusher, Cloudinary - jedná se o cloudové služby, které slouží účelům, které s~touto architekturou nejsem schopný zařídit.
		
		\begin{itemize}
			\item Pusher se stará o web sockety, respektive stále spojení, které se \uv{serverless} spojením není možné. V~Effiu posloužil pro chat v~kanálech skupin pro aktualizaci zpráv všech uživatelů pokud nějaký pošle zprávu.
			
			\item Cloudinary slouží pro ukládání obrázků a jejich distribuci do CDN.
		\end{itemize}
	\end{itemize}


\section{Autentifikace}
\subsection{Auth.js}
Auth.js je knihovna sloužící pro autentifikaci, poskytuje možnost \uv{session based}, to je použito v Effiu, a JWT autentifikace. Dále knihovna podporuje OAuth s mnohými providery, v tomto projektu je využit GitHub a Google s~jednoduchou možností přidat další. Výhodou knihovny je, že data si vývojář spravuje sám, neboli jsou ukládána do jeho vlastní databáze v podobě tabulek (které si také může sám upravit): Account, Session, User a Verification Token, které poskytují naprostou kontrolu nad ověřením uživatelů.

\subsection{Proces přihlášení}
Uživatel se na podstránce /login rozhodne zdali se přihlásit pomocí Google nebo GitHub účtu, je poté přesměrován na stránku těchto providerů, kde potvrdí přístup k~informacím o jejich účtu a je navrácen zpět do Effia. V~databázi jsou vytvořeny tabulky o tomto uživateli včetně nové relace (Session), díky které je schopen přihlášení. To proběhne automaticky po navrácení se zpět od providera.

\clearpage
\section{Model databáze}

\begin{figure}[H]
	\centering %% příkaz, který ti obrázek zarovná na střed
	\includegraphics[width=1\linewidth]{image/schema.png} 
	\caption{Databázový model Effia. Vytvořeno pomocí \href{https://github.com/Ovyerus/prismaliser}{Prismaliser}} %% popisek obrázku, nezapomeň na citace!
	\label{fig:schema} %% označení až budeš chtít na obrázek odkazovat
\end{figure}

Autorizace a autentifikace
\begin{itemize}
	\item User - tabulka s údaji o uživateli.
	\item Account - účet uživatele, typ providera přes kterého je přihlášen a data k~relaci.
	\item Verification Token - ověřovací identifikátor providera.
	\item Session - relace, do jisté míry propojuje jednotlivé tabulky.
\end{itemize} 
Otázky
\begin{itemize}
	\item Question - otázka jako taková, obsahuje název, popis, její propojení s testy atd.
	\item QuestionType - typ otázky, rozhoduje jejím chování na frontendu.
	\item QuestionRecord - záznam otázky, po vyplnění testu se vytvoří ke každé otázce jeden nový záznam s~výsledky, je sdružován Test Recordem.
\end{itemize}
Testy
\begin{itemize}
	\item Test - obsahuje název testu, jeho popis a sdružuje veškeré další podrobnosti testu jako jsou Tagy, Stars, obsahuje množství TestVersion, což je vlastně jednotlivá verze testu.
	\item TestVersion - verze testu, uchovává odpovědi, body a Mark System, verze se aktualizují při každé změně testu.
	\item TestRecord - záznam vyplněného testu, obsahuje také Question Records.
\end{itemize}
Ostatní tabulky týkající se otázek
\begin{itemize}
	\item TestStar - ohodnocení testu hvězdičkou, každý uživatel mimo majitele může test takto ohodnotit, ekvivalent "liku" na sociálních sítích
	\item MarkSystem - známkovací systém daného testu, uživatel si ho může sám upravit a při uložení testu je uchovám právě v~této tabulce.
	\item Tag/TagOnTest - uchovává štítky týkající se tématu testu vybrané majitelem.
\end{itemize}
Skupiny
\begin{itemize}
	\item Group - skupina pro uživatele, obsahuje základní vlastnosti skupiny jako jméno, slug apod.
	\item GroupSubcategory - jednotlivý kanál skupiny, každý tento kanál má i samostatné zprávy, chat apod.
	\item GroupSubcategoryMessage - Zpráva v~kanálu, vztahuje se k~ní také její odesilatel, název, obsah atd. Obsahuje také MessageType, což je druh zprávy, kterou uživatel poslal.
\end{itemize}
Ostatní
\begin{itemize}
	\item Template - šablona testu
\end{itemize}

\subsection{Cloud hosting}
Jak již bylo zmíněno v~úvodu tak tato aplikace by se měla obejít bez vlastního serveru, nejde ale jenom o databázi ale také například o ukládání obrázků nebo hostování aplikace jako takové, to je prováděno přes \uv{Cloud hostingy} jako Vercel pro hostování stránky jako takové, zároveň ale řeší i rozesílání statických dat do CDN a poskytuje serverless lambda funkce, Planetscale pro hostování mojí MySQL databáze, Cloudinary, který slouží jako \uv{bucket} pro obrázky a také jejich možnost editace přes url parametry, nebo Pusher, který slouží jako web socket server například pro chat.

\section{Využité backendové technologie}

\section{SvelteKit}

Svelte je open source JavaScriptový framework vyvíjený od roku 2016 týmem Riche Harrise, jeho hlavní výhodou je rychlost ale také intuitivita, protože jazyk se snaží vypadat jako JavaScript, zatímco rozšiřuje jeho možnosti, díky tomu se za posledních let těší rostoucí popularitě webových vývojářů. Na rozdíl od ostatních frontendových frameworků (například React, Angular, Solid nebo Qwik) Svelte disponuje vlastním jazykem, který se zapisuje do \texttt{.svelte} souboru, ten se následně kompiluje do vysoce efektivního JavaScriptu.\\

SvelteKit je metaframework postavený na Svelte. Jeho hlavní výhody se skládají z:
\begin{itemize}
	\item Rychlost - Svelte vytváří velice rychlou aplikaci, v kombinaci s Vitem (bundler) se ale také spojuje s velice rychlým build timem a hot module replacementem. To ale není jediné místo kde se rychlost projevuje, za pomocí SvelteKitu se aplikace vyvíjí velmi rychle díky velice malému množství \uv{boilerplate} kódu.
	\item Flexibilita - Aplikace často potřebuje různé typy vykreslování stránek, SvelteKit dovoluje jednoduše nakonfigurovat jednotlivé stránky či cesty pro specifické způsoby jako SPA, SSR, SSG nebo MPA. Dále také umožňuje různě kombinovat kód, který běží na serveru a ten co se spouští na klientovi, dává nad nimi rozsáhlou kontrolu a jednoduše se jeho chování upravuje.
	\item Přehlednost - SvelteKit využívá \uv{file based routing}, tedy cesty aplikace jsou generovány podle složek, které vývojář vytvoří, soubory se poté vždy jmenují stejně, \texttt{+page.svelte} pro stránku, \texttt{+layout.svelte} pro layout apod. Díky tomu je vždy jasné pro co specifický souboru slouží, přehlednosti také přidává, již u Sveltu zmíněná podobnost s JavaScriptem.
\end{itemize} 

\clearpage
Tento kód ukazuje získání dat z~databáze při načtení stránky (load funkce) a poté také actions, což jsou formulářové akce vytvářené pro specifickou cestu, které poté můžeme využívat, zde je vidět mazání uživatele ze skupiny
\begin{lstlisting}[style=ES6, caption=Ukázka z +page.server.ts, label=sveltekit_code]
export const load: ServerLoad = async (event) => {
	const users = prisma.user.findMany({ where: {name: event.params.name}})
	
	return users
}

export const actions: Actions = {
	deleteUsers: async (event) => {
		const formData = await event.request.formData()
		const users: string[] = []
		
		formData.forEach((value) => {
			users.push(value.toString())
		})
		
		try {
			
			await (await trpcServer(event)).groups.kickUsersFromGroup({
				groupSlug: event.params.name as string,
				userIds: users,
			})
			return {
				success: true
			}
		}
		catch (e) {
			if (e instanceof TRPCError) {
				return fail(getHTTPStatusCodeFromError(e), { message: e.message })
			}
			else {
				return fail(500, { message: "Something went wrong." })
			}
		}
	}
}
\end{lstlisting}

\subsection{tRPC}
V minulé sekci jsem se zmínil o problémech s otypováním API endpointů, tRPC (Typescript Remote Procedure Call) tento problém řeší tím, že vytváří dynamické typy pro jednotlivé endpointy, podle toho jak si je sami nadefinujeme, ty se potom dají volat pomocí funkcí bez přímého použití \texttt{fetche} nebo třeba \texttt{axiosu} (tyto funkce ve skutnčnosti \uv{fetch} requesty vykonávají ale programátor se o ně nemusí starat přímo), tyto funkce jsou dokonale otypované a v~kódu tím pádem fungují jako jakákoliv jiná funkce vytvořená programátorem. Další výhodou je také to, že jednotlivé \uv{procedury}, což je vlastně API endpoint, mají k~dispozici metody pro kontrolu vstupu nebo třeba middleware, který například může zjišťovat stav přihlášení uživatele, jako zbytek knihovny jsou i tyto metody perfektně otypované.
\begin{lstlisting}[style=ES6, caption=Endpoint generovaný pomocí tRPC, label=trpc_code]
getTestById: procedure.input(z.object({
	id: z.string(),
	includeGroupSubcategories: z.boolean().optional()
})).query(async ({ ctx, input }) => {
	const test = await ctx.prisma.test.findUnique({
		where: {
			id: input.id,
		},
		include: {
			subcategories: input.includeGroupSubcategories || false,
			owner: true,
			tags: {
				include: {
					tag: true
				}
			},
			testVersions: {
				include: {
					questions: {
						include: {
							type: true
						}
					}
				},
				orderBy: {
					version: "desc"
				},
				take: 1
			}
		},
	})
	
	if (!test) return null
	return test
}),
\end{lstlisting}
\clearpage
\begin{lstlisting}[style=ES6, caption=Volání funkce pomocí tRPC klienta s metodou getTestById, label=trpc_code_use]
	const imageUrlToDeleteTest = await trpc(get(page)).getTestById.query({
		id: props.data.id,
	})
\end{lstlisting}
\subsection{Prisma}
Prisma slouží jako ORM (Object–relational mapping), to znamená pomocí JavaScriptu získávat data z databáze bez přímého použití jazyka SQL, Prisma se skládá z klientské části, která pomocí protokolu založeném na JSON komunikuje se serverovou částí, tam je následně uskutečněn SQL dotaz a odpověď je poslána klientovi. Také řeší již zmíněný problém s otypováním těchto dotazů, pro Prismu je totiž nutné vytvořit \texttt{schema.prisma} soubor kde se definuje model databáze, ten poté můžeme pomocí Prisma CLI nahrávat do databáze ale také vytvářet dynamické typy, které poté využijeme jak v~dotazech tak v~aplikaci pro data, která dostaneme zpět.
Prisma je asi hlavní důvod nemožnosti využívat Edge runtime, tato knihovna je totiž příliž velká a proto pro velkou většinu mých funkcí není možné tuto architekturu využít, jako řešení by se nabízelo využít jinou knihovnu jako Drizzle ORM.
\begin{lstlisting}[style=ES6, caption=Získání testu podle id a přidání dat ze spojených tabulek, label=prisma_code_use]
	const test = await ctx.prisma.test.findUnique({
		where: {id: input.id},
		include: {
			subcategories: input.includeGroupSubcategories || false,
			owner: true,
			tags: {
				include: {tag: true}
			},
			testVersions: {
				include: {
					questions: {
						include: {
							type: true
						}
					}
				},
				orderBy: {
					version: "desc"
				},
				take: 1
			}
		},
	})
\end{lstlisting}

\clearpage
\begin{lstlisting}[style=ES6, caption=Schéma modelu verze testu, label=prisma_code_schema]
model TestVersion {
	versionId      String       @id @default(uuid())
	createdAt      DateTime     @default(now())
	updatedAt      DateTime     @updatedAt
	version        Int
	testId         String
	markSystemJSON Json
	totalPoints    Int          @default(0)
	testGroup      Test         @relation(fields: [testId], references: [id], onDelete: Cascade)
	questions      Question[]
	records        TestRecord[]
	
	@@index([testId])
	@@index([version])
}
\end{lstlisting}

\subsection{Zod}
Zod je validační knihovna jako např. Yup. To znamená, že jeho prací je kontrolovat mnou vložené vstupy, knihovna vrací úspšnost a také chyby na které během kontroly narazí. Jeho výhodou je avšak možnost využít jeho validační schémata jako typy a také samotná validace funguje jako \uv{type guard} (kontroluje a nastavuje typy u vložené proměnné). Jeho další výhodou je například jeho nízká velikost.

Tento kód kontroluje zdali vložený input odpovídá struktuře \uv{answerSchema}, popřípadě nastaví error do proměné, která se poté zobrazí klientovi.
\begin{lstlisting}[style=ES6, caption=Validace vstupu pomocí validační knihovny Zod, label=zod]
const answerSchema = z.string().min(ANSWER_MIN, `Answer has to be at least ${ANSWER_MIN} character long.`).max(ANSWER_MAX, `Answer can be max ${ANSWER_MAX} characters long.`)
const result = answerSchema.safeParse(content.answers[item].answer)
if (result.success === false) {
	isError = true
	content.answers[item].error = result.error.errors[0].message
}
\end{lstlisting}

%\section{Cloud provideři}
%Effia plně spoléhá na cloudové řešení, díky nim je jednoduché zprovoznit plně funkční stránku bez jakékoliv starosti o fyzický hardware, stejně jako o škálování zdrojů podle potřeby.
%\begin{itemize}
	%\item \textbf{Vercel} se stará o hostování aplikace jako takové, distribuci statického obsahu na %CDN a jednoduše řeší znovuvytvoření stránky po změně kódu v~GitHub repozitáři. Vercel podporuje %\uvserverless\uv architekturu v~podobě tradiční lambda funkcí. Jedná se o instance dané funkce, %která se spouští na serveru, není dedikovaná ale spouští se a vypíná podle vytížení.
%	\item \textbf{Cloudinary} slouží
%\end{itemize}

\chapter{Frontend}

\section{Design}
Jednou z~hlavních myšlenek bylo vytvořit pohledem přívětivou aplikaci, proto se návrh designu stál klíčovou částí pro stylově propracovanější prvky stránky. Pro tvorbu designu, stejně jako vytváření a úpravu potřebných obrázků jsem využil aplikaci Figma.
\begin{figure}[h]
	\centering %% příkaz, který ti obrázek zarovná na střed
	\includegraphics[width=1\linewidth]{image/figma.png} 
	\caption{Prvotní návrh domovské stránky.} %% popisek obrázku, nezapomeň na citace!
	\label{fig:figma} %% označení až budeš chtít na obrázek odkazovat
\end{figure}

\section{Responsivita}
Celá aplikace je uzpůsobená jak pro počítače tak pro mobilní zařízení. Responsivita není vůbec lehká práce, mně ale pomohl Tailwind, ve kterém se CSS media query dělají snadněji společně s moderními CSS containers, které umožňují responsivní breakpointy odvozovat ne jen od velikosti stránky ale od rozměrů rodičovských elementů.

Problematika však nespočívá pouze v~zobrazení prvků ale také v~jejich funkcionalitě, která musí fungovat jak s~myší, tak s~dotekem, příkladem tohoto problému je například jeden z~typů otázek, a to \uv{Connect}, kde se musí jednotlivé spoje přesouvat jak pohybem myši tak při dotykovém vstupu.

\begin{figure}[h]
	\centering
	\begin{minipage}[]{0.49\textwidth}
		\centering
		\includegraphics[width=0.5\linewidth]{image/res1.png}
		\caption{Domovská stránka na mobilním zařízení}
		\label{fig:res1}
	\end{minipage}
	\hfill
	\begin{minipage}[]{0.49\textwidth}
		\centering
		\includegraphics[width=0.5\linewidth]{image/res2.png}
		\caption{Generátor testů na mobilním zařízení}
		\label{fig:res2}
	\end{minipage}
\end{figure}

\section{Světlý a tmavý režim}
S ohledem na uživatele co preferují tmavý režim jsem se také rozhodl pro tvorbu tmavého režimu, ten je možné vidět na obrázku \ref{fig:test-creator2}, oba tyto režimy vyžadovali vlastní paletu barev a byly mnohokrát přepracovány aby k~sobě jednotlivé barvy co nejlépe pasovaly.

\section{Využité frontendové technologie}
\subsection{Svelte}
Svelte je open source JavaScriptový framework vyvíjený od roku týmem Riche Harrise, jeho hlavní výhodou je rychlost ale také intuitivita, protože jazyk se snaží vypadat jako JavaScript, zatímco rozšiřuje jeho možnosti, díky tomu se za posledních let těší rostoucí popularitě webobých vývojářů. Na rozdíl od ostatních webových frameworků (například React, Angular, Solid nebo Qwik) Svelte disponuje vlastním jazykem, který se zapisuje do \texttt{.svelte} souboru, ten se následně kompiluje do vysoce efektivního JavaScriptu. Soubor jako samotný velice připomíná HTML, které se ale dělí na určité prvky.

A to přesněji na 3:
\begin{itemize}
	\item Script - jedná se o část, kde se vypisují funkce, proměnné a řeší se reaktivní deklarace. Celá tato část je obklopená \texttt{<script>} tagy jako v~HTML dokumentu.
	\begin{lstlisting}[style=ES6, caption=Ukázka Svelte kódu ve script tagu, label=svelte-script-sample]
	export let inputValue: HTMLTextAreaElement['value'] = '';
	
	let setError = getContext('setError');
	
	let inputRef: HTMLTextAreaElement;
	
	const dispatch = createEventDispatcher();
	
	function validateInput() {
		const result = validationSchema?.safeParse(inputValue);
		if (!result?.success) {
			dispatch('error', result?.error.errors[0].message);
			if (typeof setError === 'function')
			setError(result?.error.errors[0].message);
		} else {
			dispatch('error', null);
			if (typeof setError === 'function') setError('');
		}
	}
	
	function dispatchInputChange() {
		dispatch('inputChange', inputRef.value);
	}
	\end{lstlisting}
	\item Style - tato část slouží jako CSS pro daný dokument, výhodou je ale lokální rozsah aplikovaných stylů (jako u CSS modules ale mnohem lépe provedené), aplikovat zde lze i globální styly díky \texttt{:global}. Osobně mi vyhovuje, že se styly nacházejí v~jednom souboru, a i když jsem tento způsob pro stylování Effia převážně nevyužíval, tak osobně mi připadá velice dobře provedený. Celý úsek tohoto kódu je jako v~HTML dokumentu formován do \texttt{<style>} tagů.
	\clearpage
	\begin{lstlisting}[style=ES6, caption=Ukázka Svelte CSS kódu, label=svelte-CSS-sample]
	<style>
		.grid_cover {
			display: grid;
			grid-template-rows: auto 1fr;
		}
		:global(.dark) .fading {
			background-color: var(--dark-light_grey);
		}
	</style>
	\end{lstlisting}
	\item Zbytek kódu - vše co se nenachází v~jednom z~těchto tagů je HTML reprezentace dané stránky, nejedná se však o čisté HTML ale jeho obohacenou verzi. Jako v~podobných frameworcích je možné přidávat \uv{event listenery} na jednotlivé elementy, podmínkově zobrazovat, pracovat s~asynchroním kódem nebo dokonce \uv{svazovat} element či hodnotu elementu s~proměnou ve scriptové části.
	\clearpage
	\begin{lstlisting}[style=ES6, caption=Ukázka Svelte kódu, label=svelte-sample]
	{#await data}
		<div class="@container h-full">
		<div class="flex w-full py-1 scroller flex-nowrap">
			{#each Array(countOfItems).fill('') as _}
				<div
				class="min-w-[calc(100%/var(--items-count))] h-full relative aspect-[4/5]"
				>
					<!-- Zde se nachází další kód -->
				</div>
			{/each}
			<span class="loading loading-infinity loading-lg" />
		</div>
	</div>
	{:then awaitedData}
		<div class="@container h-full">
			<div
			bind:this={scrollerDiv}
			class="flex w-full h-full py-1 scroller flex-nowrap"
			style="--translate-x: 0%;"
			>
				{#each awaitedData as item}
					<div
					class="min-w-[calc(100%/var(--items-count))] relative aspect-[4/5]"
					>
						<CardAlternative
						class="mx-auto"
						navigationLink={'/tests/' + item.id}
						type={item.type}
						data={{
								...item
						}}
						/>
					</div>
				{/each}
			</div>
		</div>
	{/await}
	\end{lstlisting}
\end{itemize}
\clearpage
\subsection{Typescript}
Typescript se dá považovat jako nadstavba JavaScriptu, poskytuje ale jednu výraznou výhodu - typy, díky nim je možno mnohem snadněji dohledávat chyby, vracet se k dříve napsanému kódu a celkově mnohem zlepší \uv{developer experience} při vytváření aplikace. Sám o sobě pomůže s otypováním jednotlivých částí kódu, neporadí si však například s API endpointy nebo databázovými dotazy, které se poté musí otypovat ručně, což je ale velice špatný způsob. Proto se v~tomto projektu využívají další knihovny jako Prisma, tRPC a Zod.

\begin{lstlisting}[style=ES6, caption=Ukázka TypeScriptového typu, label=typescript-sample]
export type IdCardAlternativeProps = CardAlternativeProps & {
	id: string;
	type: TestType;
};

export type CarouselItemInput =
| IdCardAlternativeProps[]
| Promise<IdCardAlternativeProps[]>;
\end{lstlisting}

\subsection{Tailwind CSS}
Tailwind CSS je \uv{CSS utility library}, to znamená, že narozdíl od frameworků jako je třeba Bootstrap nebo Material UI neposkytuje celé předpřipravené komponenty ale připravené CSS classy, které se aplikují na HTML elementy, jeho výhodou je naprostá kontrola nad chováním stylů, které se aplikují, přehlednost a rychlost se kterou se dá styly vytvářet.
\begin{lstlisting}[style=ES6, caption=Ukázka Tailwind kódu, label=tailwind-sample]
<button
	type="button"
	on:click={starTest}
	disabled={canStarTest === false || isSubmittingStar === true}
	class={`absolute flex items-center z-[2] gap-1 px-2 py-1 rounded-lg right-1 top-1 bg-light_white dark:bg-dark_grey shadow-md duration-100 ${
			isStarred ? 'bg-yellow-100 dark:bg-yellow-700' : ''
		} hover:bg-light_secondary dark:hover:bg-dark_secondary disabled:bg-light_grey_dark dark:disabled:bg-slate-600
		text-light_text_black dark:text-dark_text_white hover:text-light_whiter disabled:hover:text-light_text_black
		dark:disabled:hover:text-dark_text_white`}
	>
</button>
\end{lstlisting}

\chapter{Funkcionality aplikace a implementace}

\section{Domovská stránka}
Domovská stránka slouží jako místo pro seznámení návštěvníka s~výhodami Effia, rychlá navigace mezi jednotlivými stránkami ale také jako místo nejpečlivěji vytvářeného designu a efektů aby na uživateli zanechala dojem kvalitní aplikace.

\begin{figure}[h]
	\centering %% příkaz, který ti obrázek zarovná na střed
	\includegraphics[width=1\linewidth]{image/homepage-notlogged.png} 
	\caption{Domovská stránka Effia pro nepřihlášeného uživatele.} %% popisek obrázku, nezapomeň na citace!
	\label{fig:homepage} %% označení až budeš chtít na obrázek odkazovat
\end{figure}
\clearpage

\section{Přihlášení}
Po kliknutí na tlačítko login se dostaneme na přihlašovací obrazovku kde si můžeme vybrat mezi přihlášením přes Google a GitHub účet, po úspěšném přihlášení se uživatel dostane do dashboardu, což je společně s~vytvářením testů, historií a skupinami dostupné pouze pro přihlášené uživatele, pokus načtení stránky pokud je uživatel nepřihlášení vyústí v~přesměrování na login page.

\begin{figure}[h]
	\centering %% příkaz, který ti obrázek zarovná na střed
	\includegraphics[width=1\linewidth]{image/homepage.png} 
	\caption{Domovská stránka Effia pro přihlášeného uživatele.} %% popisek obrázku, nezapomeň na citace!
	\label{fig:homepage} %% označení až budeš chtít na obrázek odkazovat
\end{figure}

\section{Testy a jejich vlastnosti}
Jednou z esenciálních funkcionalit Effia je možnost vytvořit test, k~tomuto existuje mnoho různých postupů, nad danou problematikou jsem se zprvu zamýšlel a až poté napsal funkční nástroj pro jejich vytváření ale to stejně nakonec nezabránilo následné nutnosti přepsat téměř celou funkcionalitu při vytváření testu.

\subsection{Tvorba testu}
\label{subsec:creation}
Jako první si uživatel zvolí mezi kvízovým a programovacím testem, u obou si poté vybere šablonu.
\begin{itemize}
	\item Kvízový - po výběru šablony, kde si uživatel může zvolit i import z GIFT formátu, se uživatel dostane do tvorby testu samotného, vybírat si aktuálně může z~6~typů otázek: \textit{Pick One}, \textit{True/False}, \textit{Connect}, \textit{Write}, \textit{Fill} a \textit{Geography}, otázkám lze svévolně měnit pořadí, přidávat komentáře k~odpovědím a upravovat počet získaných bodů.
	\item Programovací - po výběru šablony se uživatel dostane do tvorby testu programovacího, kde ho pojmenuje problém, popíše co má uživatel řešit, nadefinuje kontrolní vstupy a očekávané výstupy, poté může zanechat nápovědy
\end{itemize}
Po dokončení těchto úprav se uživatel dostane do konečných úprav testu, což činí jméno, popisek a obrázek testu, volitelné zařazení do skupin, tagy, rozhodne se jestli využít známkovací systém, který si může sám upravit, zvolí si zdali náhodně třídit otázky a následně tvorbu ukončí a rozhodne se zdali test uložit jako návrh nebo ho publikovat.

\begin{figure}[h]
	\centering
	\begin{minipage}[]{0.49\textwidth}
		\includegraphics[width=\textwidth]{image/test-creator1.png}
		\caption{Kvízový test}
		\label{fig:test-creator1}
	\end{minipage}
	\hfill
	\begin{minipage}[]{0.49\textwidth}
		\includegraphics[width=\textwidth]{image/test-creator2.png}
		\caption{Programovací test}
		\label{fig:test-creator2}
	\end{minipage}
\end{figure}

\section{Vyplňování testu}
\label{sec:test-take}
\subsection{Vyplňování kvízu}
Vytvořený test si poté může kdokoliv s~přístupem k~němu vyplnit (testy jsou základně dostupné pro všechny, po úpravě mohou být zveřejněny pouze pro členy skupin).

Otázky se náhodně seřadí a uživatel je vyplňuje, zamíchané jsou také odpovědi určitých typů otázek. Po vyplnění všech uživatel test odevzdá, zkontroluje se a vrátí mu správné odpovědi, počet bodů, známku co získal a pokud je uživatel přihlášený tak se záznam a vyplnění uloží do databáze, uživatel si ho poté může zpětně zobrazit v sekci \textit{Test history}.

\begin{figure}[H]
	\centering %% příkaz, který ti obrázek zarovná na střed
	\includegraphics[width=0.75\linewidth]{image/test-taking.png} 
	\caption{Obrázek kvízu.} %% popisek obrázku, nezapomeň na citace!
	\label{fig:test-taking} %% označení až budeš chtít na obrázek odkazovat
\end{figure}

\subsection{Plnění programovacího testu}
Uživatel dostane popis toho co by měl kód umět, sadu testů, které mají otestovat funkcionalitu kódu a popřípadě nějaké nápovědy. Programovací test obsahuje vlastní editor do kterého uživatel píše, pro kontrolu testu můžeme použít tlačítko \textit{\uv{Run}} a pokud testy prochází tak je test možné řešení odevzdat.

\begin{figure}[H]
	\centering %% příkaz, který ti obrázek zarovná na střed
	\includegraphics[width=0.75\linewidth]{image/programming.png} 
	\caption{Obrázek programovací úlohy.} %% popisek obrázku, nezapomeň na citace!
	\label{fig:programming} %% označení až budeš chtít na obrázek odkazovat
\end{figure}

\clearpage
\section{Zobrazování testů}

\subsection{Komunitní místo}
\label{subsec:community}
Na této stránce může uživatel najít nové a populární testy včetně všech testů, které existují. Testy jsou zobrazovány postupně podle techniky \uv{infinite scrolling}, na což využívám JavaScript API Intersection Observer, abych zjistil, kdy uživatel dosáhl posledního prvku a vyžádal si tak další testy. Implementované je vyhledávací pole, které filtruje zobrazované testy, další možností je filtrace pomocí tagů. Vizuálně jsou také rozlišeny kvízy od programovacích testů. Testy se dají ohodnotit hvězdičkou, jejich aplikování využívá principu \uv{optimistic update}, to znamená, že po přidání hvězdičky ji uživatel okamžitě vidí přidanou, zatímco se ověřuje jeho oprávnění a vytváří záznam hvězdičky v databázi. V~případě neúspěchu se poté uživateli sama hvězdička opět odebere. 

\begin{figure}[H]
	\centering %% příkaz, který ti obrázek zarovná na střed
	\includegraphics[width=1\linewidth]{image/community.png} 
	\caption{Komunitní místo.} %% popisek obrázku, nezapomeň na citace!
	\label{fig:community} %% označení až budeš chtít na obrázek odkazovat
\end{figure}


\subsection{Kolekce testů}
\label{subsec:collection}
Zde si uživatel může zobrazit jím vytvořené testy, aplikovaná je stejná funkcionalita vyhledávacího pole a \uv{infinite scrollingu}. Každý test má ale také další možnosti, a to úpravu, export a smazání.
\begin{itemize}
	\item Úprava - uživatel se přesune na stránku úprav, tam může celý test přepracovat.
	\item Export - vytvoří z~testu textový soubor ve formátu GIFT se všemi otázkami daného testu, které jsou podporovány Moodlem
	\item Delete - smazání testu z databáze
\end{itemize}


\section{Skupiny}
Každý přihlášený uživatel si může vytvořit vlastní skupinu, do které se můžou pomocí generovaného kódu připojit ostatní uživatelé. Skupina obsahuje kanály, ve kterých je možné psát textové zprávy, taky zde můžeme nají přidané testy, vlastník si poté může procházet grafy výsledku členů skupiny.

%\chapter{Výsledky práce}
%\section{Funkce aplikace}
%Po příchodu na stránku se uživateli zobrazí domovská stránka, zde může prozkoumat výhody aplikace nebo se přihlásit, to ho přesune na přihlašovací stránku, na ni se uživatel do aplikace může přihlásit pomocí Google nebo GitHub účtu. Po přihlášení je uživatel přesunut na dashboard, kde může vidět rychlou navigaci na další části nebo se podívat na souhrn jeho aktivity v~grafech. První možností je vytvořit si nový test, tato možnost je detailně probraná zde \ref{subsec:creation}.
%Po vytvoření testu je přesunut do kolekce jeho testů, kde může své testy procházet a upravovat, více popsáno zde \ref{subsec:collection}. Test si poté může uživatel zkusit vyplnit \ref{sec:test-take}, buď přímo ze své kolekce nebo z komunitního centra \ref{subsec:community}, kde může najít také testy jiných. Po dokončení testu se dozví výsledky, ty může poté najít zpětně v~sekci \textit{Test history}, kde jsou uspořádané v tabulce.

\chapter{Zhodnocení práce}

\section{Splněné a nesplněné cíle}
Cíle byly rozdělené jak do rozsahu aplikace tak do kvality implementace jednotlivých prvků, co se týče rozsahu tak ten jsem v~určitých místech významně předčil, ne všechny body původních cílů jsou ale aktuálně plně dosaženy.

Hlavním cílem bylo vytvořit rychlou, cloudovou aplikaci pro vytváření a sdílení testů využívající moderních \uv{techstack}, v~tomto ohledu jsem cíl kompletně splnil, aplikace je v~těchto bodech plně funkční a můj původní plán využití technologií se během vývoje ještě významně rozrostl.

Za úspěch považuji také grafickou část aplikace, která za mě tvoří minimalistický moderní vzhled.

Hlavní neúspěch nebo spíše nedodělanost vidím ve skupinách a přizpůsobení možností testů pro ně, původně jsem zamýšlel vytvořit skupiny jako místo pro sdílení materiálů s více zajímavými možnostmi pro vlastníka, aby sloužili jako funkce vhodná pro výuku. Cíle nejsou zdaleka nerealistické, ale těchto cílů jsem nebyl schopen dosáhnou z~nedostatku času, jako vylepšení by ale za mně bylo velice přínosné.
	
Další částečný neúspěch vidím v programovacích testech, ty se i přes snahu vyladit nepovedly dostat do stavu kdy by byly pro uživatele nezávadné, v~aktuálním stavu je naprosto možné přetížit vlastní prohlížeč napsáním např. \texttt{while(true) {}}, JavaScript je \uv{single threaded} jazyk a proto není možné proces ukončit, protože neskončí ten předchozí (právě zmínění while loop). Řešení nabízí \uv{Web workers}, což je možnost jak JavaScript spouštět na více vláknech, zdálo se tedy, že řešení je na světě. Po bližším přezkoumání jsem ale zjistil, že nejsem schopen web workery v~SvelteKit aplikaci v~produkci načítat žádným způsobem, daním web workeru do stejné složky dojde k~neexistující referenci (web worker se zbundluje se zbytkem kódu a jeho funkcionalita se rozbije), řešení je tedy dát worker do public složky, která se odděluje od zbytku, zde ale dojde k~erroru MIME.
\chapter*{Závěr}
	
Cílem projektu bylo vytvořit webovou aplikaci pro vytváření a vyplňování testů. Aplikace je postavená na frameworku SvelteKit, psaný v jazyce TypeScript. Přihlašování stojí na knihovně Auth.js, MySQL databáze je hostovaná službou Planetscale, jako ORM je použita Prisma. Frontend řeší framework Svelte s~CSS utility knihovnou Tailwind, pro validaci využívá Zod.

Základem aplikace je generátor testů, kde uživatel využívá předpřipravené typy otázek. Testy se následně dají vyplnit v rámci komunity nebo vlastníkovi kolekce. Nepřihlášený uživatel může testy pouze vyplňovat, přihlásit se uživatel může pomocí Google nebo GitHub účtu. Uživatelé také mají k~dispozici skupiny kde je chat a také možnosti sdílet testy, test historii k~zobrazení dříve vyplněných testů a přehled nedávné aktivity v podobě grafů. Využité technologie činí aplikaci lehce škálovatelnou a velice výkonnou. Aplikace je téměř zcela funkční a použitelná na všech zařízeních díky responzivnímu designu.

\noindent Aplikace je zálohovaná na GitHubu na adrese \url{https://github.com/matej-kotrba/effio}.
Je také volně přístupná na adrese \url{https://effio.vercel.app/}
	
	%% literatura
	\begin{thebibliography}{99}
		
		\bibitem{Svelte} Svelte [online]. 2023 [cit. 2023-12-27]. Dostupné z: \url{https://svelte.dev/}
		\bibitem{SvelteKit} SvelteKit [online]. 2023 [cit. 2023-12-27]. Dostupné z:  \url{https://kit.svelte.dev/}
		\bibitem{tRPC} tRPC [online]. 2023 [cit. 2023-12-27]. Dostupné z: \url{https://trpc.io/}
		\bibitem{Prisma} Prisma [online]. 2023 [cit. 2023-12-27]. Dostupné z: \url{https://www.prisma.io/}
		\bibitem{Zod} Zod [online]. 2023 [cit. 2023-12-27]. Dostupné z: \url{https://zod.dev/}
		\bibitem{Auth.js} Auth.js [online]. 2023 [cit. 2023-12-27]. Dostupné z: \url{https://authjs.dev/}
		\bibitem{Tailwind CSS} Tailwind CSS [online]. 2023 [cit. 2023-12-27]. Dostupné z: \url{https://tailwindcss.com/}
		\bibitem{tRPC-SvelteKit} tRPC-SvelteKit [online]. 2023 [cit. 2023-12-27]. Dostupné z: \url{https://icflorescu.github.io/trpc-sveltekit/}
		\bibitem{ChatGPT} ChatGPT [online]. 2023 [cit. 2023-12-27]. Dostupné z: \url{https://chat.openai.com/}
		\bibitem{Stackoverflow} Stackoverflow [online]. 2023 [cit. 2023-12-27]. Dostupné z: \url{https://stackoverflow.com/}
		\bibitem{Joy of Code}Joy of Code. Youtube kanál. Https://www.youtube.com/ [online]. 2023 [cit. 2023-12-27]. Dostupné z: \url{https://www.youtube.com/@JoyofCodeDev}
		\bibitem{Huntabyte}Huntabyte. Youtube kanál. Https://www.youtube.com/ [online]. 2023 [cit. 2023-12-27]. Dostupné z: \url{https://www.youtube.com/@Huntabyte}
		\bibitem{Theo}BROWNE, Theo. Youtube kanál. Https://www.youtube.com/ [online]. 2023 [cit. 2023-12-27]. Dostupné z: \url{https://www.youtube.com/@t3dotgg}
		\bibitem{Kevin Powell}POWELL, Kevin. Youtube kanál. Https://www.youtube.com/ [online]. 2023 [cit. 2023-12-27]. Dostupné z: \url{https://www.youtube.com/@KevinPowell}
		\
		
	\end{thebibliography}
	
	%% obrázky 
	\listoffigures
	

	\appendix %% začínají přílohy
	
	\titleformat{\chapter}[block]{\scshape\bfseries\LARGE}{Příloha \thechapter}{10pt}{\vspace{0pt}}[\vspace{-22pt}] %% nastavení nadpisu u příloh
	

	
\end{document}